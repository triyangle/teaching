\Question{Univariate LASSO}
In this problem, we will derive the closed form solution to the univariate LASSO problem:

 \[
     \min_{w} \frac{1}{2}\|wx - y\|^{2}_{2} + \lambda \|w\|_{1}
.\]


\begin{Parts}

    \Part Considering the scenario with $w, x, y, \lambda \in \mathbb{R}$ and
    $\lambda \geq 0$, rewrite the problem as the sum of two convex functions of
    $w$. Note that $x^{2}$ and $|x|$ are both convex functions of $x$.

\begin{solution}
\[
    \min_{w \in \mathbb{R}} \frac{1}{2} \left( wx - y \right)^{2} + \lambda |w|
.\] 
\end{solution}

\Part Find the derivative of the objective function from part (a) with respect
to $w$. Recall that
\[
    \frac{d}{dx} |x| = \begin{cases}
        1, & \text{if } x > 0 \\
        -1, & \text{if } x < 0 \\
    \text{undefined}, & \text{if } x = 0 \end{cases}
.\] 

\begin{solution}
    \begin{align*}
         \frac{d}{d w} \left( \frac{1}{2} \left( wx - y \right)^{2} + \lambda
         |w| \right) &= \left( wx - y \right)x + \begin{cases}
             \lambda, & \text{if }  w > 0 \\
             -\lambda, & \text{if } w < 0 \\
             \text{undefined}, & \text{if } w = 0
         \end{cases}
         \\
                               &= wx^{2} - xy + \begin{cases}
             \lambda, & \text{if }  w > 0 \\
             -\lambda, & \text{if } w < 0 \\
             \text{undefined}, & \text{if } w = 0
         \end{cases}
     \end{align*}
\end{solution}

\Part Draw the graph of the derivative as a function of $w$ from part (b) for each of the cases
below. Recall that $\lambda \geq 0$.
\begin{enumerate}[i.]
    \item $-xy + \lambda < 0$

    \item $|-xy| \leq \lambda$

    \item $-xy - \lambda > 0$
\end{enumerate}

\begin{solution}
    \begin{enumerate}[i.]
        \item Sample graph for $x^{2} = 1$, $-xy = -2$, $\lambda = 1$

            %\pgfplotsset{compat=1.8}
            \begin{tikzpicture}[]
                \begin{axis}[
                    axis x line=middle, axis y line=middle,
                    ymin=-5, ymax=5, ytick={-5,...,5}, ylabel=$\frac{d}{dw}$,
                    xmin=-5, xmax=5, xtick={-5,...,5}, xlabel=$w$,
                    domain=-pi:pi,samples=101, % added
                    ]

                    \addplot [domain=-5:0,blue] {x - 2 - 1};
                    \addplot [domain=0:5,blue] {x - 2 + 1};

                    \draw [draw=blue, fill=white, thick] (axis cs: 0, -1) circle (2.0pt);
                    \draw [draw=blue, fill=white, thick] (axis cs: 0, -3) circle (2.0pt);
                \end{axis}
            \end{tikzpicture} 

        \item Sample graph for $x^{2} = 1$, $-xy = -1$, $\lambda = 2$

            %\pgfplotsset{compat=1.8}
            \begin{tikzpicture}[]
                \begin{axis}[
                    axis x line=middle, axis y line=middle,
                    ymin=-5, ymax=5, ytick={-5,...,5}, ylabel=$\frac{d}{dw}$,
                    xmin=-5, xmax=5, xtick={-5,...,5}, xlabel=$w$,
                    domain=-pi:pi,samples=101, % added
                    ]

                    \addplot [domain=-5:0,blue] {x - 1 - 2};
                    \addplot [domain=0:5,blue] {x - 1 + 2};

                    \draw [draw=blue, fill=white, thick] (axis cs: 0, 1) circle (2.0pt);
                    \draw [draw=blue, fill=white, thick] (axis cs: 0, -3) circle (2.0pt);
                \end{axis}
            \end{tikzpicture} 

        \item Sample graph for $x^{2} = 1$, $-xy = 2$, $\lambda = 1$

            %\pgfplotsset{compat=1.8}
            \begin{tikzpicture}[]
                \begin{axis}[
                    axis x line=middle, axis y line=middle,
                    ymin=-5, ymax=5, ytick={-5,...,5}, ylabel=$\frac{d}{dw}$,
                    xmin=-5, xmax=5, xtick={-5,...,5}, xlabel=$w$,
                    domain=-pi:pi,samples=101, % added
                    ]

                    \addplot [domain=-5:0,blue] {x + 2 - 1};
                    \addplot [domain=0:5,blue] {x + 2 + 1};

                    \draw [draw=blue, fill=white, thick] (axis cs: 0, 1) circle (2.0pt);
                    \draw [draw=blue, fill=white, thick] (axis cs: 0, 3) circle (2.0pt);
                \end{axis}
            \end{tikzpicture} 
    \end{enumerate}
\end{solution}

\Part From the graphs and cases in part (c), deduce the closed form solution to
the univariate LASSO problem.

\begin{solution}
    In the case $-xy + \lambda < 0$, we can see from the graph in part (c)i.
    that the objective function is decreasing up to the stationary point where the derivative
    is 0. Moreover, this occurs in the linear part of the derivative where $w >
    0$. Setting the derivative to $0$ and solving for $w$ in this case, we get that
    \begin{align*}
        wx^{2} - xy + \lambda &= 0 \\
        w &= \frac{xy - \lambda}{x^{2}}
    \end{align*}

    In the case $|-xy| \leq \lambda$, the derivative never takes on the value
    $0$, and instead the minimum must occur at the discontinuity $w = 0$ as the
    objective function shifts from decreasing to increasing at that point.

    In the case $-xy - \lambda > 0$, we see again that the function has a
    stationary point for which the minimum value occurs since it is increasing
    beyond that point. This occurs in the linear part of the derivative where
    $w < 0$. Setting the derivative to $0$ and solving for $w$ in this case, we
    get that
    \begin{align*}
        wx^{2} - xy - \lambda = 0 \\
        w &= \frac{xy + \lambda}{x^{2}}
    \end{align*}

    Thus, the closed form solution to the univariate LASSO problem is:
    \[
        w = \begin{cases}
            \frac{xy - \lambda}{x^{2}}, & \text{if } -xy + \lambda < 0 \\
            0, & \text{if } |-xy| \leq \lambda \\
            \frac{xy + \lambda}{x^{2}}, & \text{if } -xy - \lambda > 0
        \end{cases}
    .\] 

    Note in particular that $\lambda$ acts as a threshold for whether $w = 0$
    or not. In particular, we see that larger values of $\lambda$ relaxes the
    condition for when $w = 0$, so that it is more conducive to setting $w =
    0$. This is consistent with how LASSO regularization encourages sparse
    solution weight vectors and how increasing $\lambda$ causes the minimizer
    to tend toward sparser solutions.
\end{solution}

\end{Parts}
